
% Included in 03_test.Rmd using 
% Included in 03_test.Rmd using 
% Included in 03_test.Rmd using 
% Included in 03_test.Rmd using \input{content/04_pure_latex}

\section{Use of pure \LaTeX}

You can integrate a \LaTeX{} file into this report.
To do this, include it with the command \texttt{\textbackslash{}input\{content/filename.tex\}} into one of your (R)md-documents.

\begin{quote}
   \textbf{Note,} however, that these \TeX documents are not tracked for changes
   from your \texttt{Makefile}.
\end{quote}

!`\TeX\ can g\`{e}n\'{e}rate alm\"{o}st all the accents\footnote{This is an
example of a footnote.} and spe\c{c}ial symbols used in Western \cite{paper2}
l\aa nguages! Likewise, its arsenal of mathematical sym$\beta$ols, introduced
below, is formidable.

\subsubsection{Mathematical Formulae}
\TeX\ is good at typesetting mathematical formulas like
       $ x-3y = 7 $
or
       $ y_{i+1} = x_{i}^{2n} - \sqrt{5}x_{i}^{n} + 1$.
Remember that a letter like
       \( x \)        % $ ... $  and  \( ... \)  are equivalent
is a formula when it denotes a mathematical symbol, and should
be treated as one.

Mathematical formulas may also be       %  use \[ \]
{\em displayed}.  A displayed formula       %  or  $$  $$
is one-line long; multiline formulas        %  or \begin{equation}
require special formatting instructions.    %  \begin{eqnarray}
The following formulae demonstrate
many constructions you might find useful.
Refer to equation (\ref{eq:fermat}), which is probably true,
while equations (\ref{eq:dumb}-\ref{eq:realdumb}) are silly.
Note that the \verb9equation9 and \verb9eqnarray9 environments
number the equations, but \verb9eqnarray*9 doesn't.
\[  x_{i+1} ~=~ N^{i+1}(x_{0}) ~=~ N(x_{i}) ~=~
    x_{i} - \frac{f(x_{i})}{f'(x_{i})}  \]

$$ \frac{\partial u}{\partial t} + \nabla^{4}u + \nabla^{2}u +
    \frac12    |\nabla u|^{2}~ =~ c^2   $$

\begin{equation}  a^{p} + b^p   \neq c^{p} ~~~\mbox{for } p>2
    ~~ \mbox{(see proof in margin)}  \label{eq:fermat}
\end{equation}
$$ \lim_{n \rightarrow \infty}x_{n} \geq \pi $$
$$ \forall x \in {\cal O} ~~\exists \delta ~~~\mbox{such that}~~~
    |y-x|<\delta ~\Rightarrow ~y \in {\cal O} $$
\vspace{2mm}
$$
\Psi' = \frac{d}{d \phi} \left( \begin{array}{c}
  \phi_{2}  \\  \phi_{3}  \\  1 - \phi_{2} - \phi_{1}^{2}/2
 \end{array} \right)
 ~~~~~~~~~~~~~
 \Theta =  \left(   \begin{array}{ccc}
    0           &   1   &   0   \\
- \theta_{1} \psi_{1} - \psi_{2} &  0   &   \psi_3  \\
    -\phi_{1}           &   -1  &   0
            \end{array}  \right)
$$
\vspace{2mm}
\begin{eqnarray}
    \int_0^{\infty} e^{-x^2}\,dx
    & = &   e^{-\left(\int_0^{\infty}x\,dx\right)^2} \label{eq:dumb}  \\
    & = &   e^{-\infty} ~~~~~\mbox{(bogus)} \\
    & = &   0.38-1.7i ~~~~~\mbox{(not!)} \label{eq:realdumb}
\end{eqnarray}
\vspace{2mm}
\begin{eqnarray*}           %  "*" = no line numbering
  \sum_{n=1}^k \frac1n
    & \approx & \ln k + \gamma  \\
    & = &       (\ln 10)(\log_{10}k) + \gamma \\
    & \approx & 2.3026\log_{10}k + 0.57772
\end{eqnarray*}

Unary operators ``plus'' and ``minus'' -- just use exponentiation:
$$  {}^{+}0.168  \mbox{ or } {}^{-}1.168    $$


\section{Use of pure \LaTeX}

You can integrate a \LaTeX{} file into this report.
To do this, include it with the command \texttt{\textbackslash{}input\{content/filename.tex\}} into one of your (R)md-documents.

\begin{quote}
   \textbf{Note,} however, that these \TeX documents are not tracked for changes
   from your \texttt{Makefile}.
\end{quote}

!`\TeX\ can g\`{e}n\'{e}rate alm\"{o}st all the accents\footnote{This is an
example of a footnote.} and spe\c{c}ial symbols used in Western \cite{paper2}
l\aa nguages! Likewise, its arsenal of mathematical sym$\beta$ols, introduced
below, is formidable.

\subsubsection{Mathematical Formulae}
\TeX\ is good at typesetting mathematical formulas like
       $ x-3y = 7 $
or
       $ y_{i+1} = x_{i}^{2n} - \sqrt{5}x_{i}^{n} + 1$.
Remember that a letter like
       \( x \)        % $ ... $  and  \( ... \)  are equivalent
is a formula when it denotes a mathematical symbol, and should
be treated as one.

Mathematical formulas may also be       %  use \[ \]
{\em displayed}.  A displayed formula       %  or  $$  $$
is one-line long; multiline formulas        %  or \begin{equation}
require special formatting instructions.    %  \begin{eqnarray}
The following formulae demonstrate
many constructions you might find useful.
Refer to equation (\ref{eq:fermat}), which is probably true,
while equations (\ref{eq:dumb}-\ref{eq:realdumb}) are silly.
Note that the \verb9equation9 and \verb9eqnarray9 environments
number the equations, but \verb9eqnarray*9 doesn't.
\[  x_{i+1} ~=~ N^{i+1}(x_{0}) ~=~ N(x_{i}) ~=~
    x_{i} - \frac{f(x_{i})}{f'(x_{i})}  \]

$$ \frac{\partial u}{\partial t} + \nabla^{4}u + \nabla^{2}u +
    \frac12    |\nabla u|^{2}~ =~ c^2   $$

\begin{equation}  a^{p} + b^p   \neq c^{p} ~~~\mbox{for } p>2
    ~~ \mbox{(see proof in margin)}  \label{eq:fermat}
\end{equation}
$$ \lim_{n \rightarrow \infty}x_{n} \geq \pi $$
$$ \forall x \in {\cal O} ~~\exists \delta ~~~\mbox{such that}~~~
    |y-x|<\delta ~\Rightarrow ~y \in {\cal O} $$
\vspace{2mm}
$$
\Psi' = \frac{d}{d \phi} \left( \begin{array}{c}
  \phi_{2}  \\  \phi_{3}  \\  1 - \phi_{2} - \phi_{1}^{2}/2
 \end{array} \right)
 ~~~~~~~~~~~~~
 \Theta =  \left(   \begin{array}{ccc}
    0           &   1   &   0   \\
- \theta_{1} \psi_{1} - \psi_{2} &  0   &   \psi_3  \\
    -\phi_{1}           &   -1  &   0
            \end{array}  \right)
$$
\vspace{2mm}
\begin{eqnarray}
    \int_0^{\infty} e^{-x^2}\,dx
    & = &   e^{-\left(\int_0^{\infty}x\,dx\right)^2} \label{eq:dumb}  \\
    & = &   e^{-\infty} ~~~~~\mbox{(bogus)} \\
    & = &   0.38-1.7i ~~~~~\mbox{(not!)} \label{eq:realdumb}
\end{eqnarray}
\vspace{2mm}
\begin{eqnarray*}           %  "*" = no line numbering
  \sum_{n=1}^k \frac1n
    & \approx & \ln k + \gamma  \\
    & = &       (\ln 10)(\log_{10}k) + \gamma \\
    & \approx & 2.3026\log_{10}k + 0.57772
\end{eqnarray*}

Unary operators ``plus'' and ``minus'' -- just use exponentiation:
$$  {}^{+}0.168  \mbox{ or } {}^{-}1.168    $$


\section{Use of pure \LaTeX}

You can integrate a \LaTeX{} file into this report.
To do this, include it with the command \texttt{\textbackslash{}input\{content/filename.tex\}} into one of your (R)md-documents.

\begin{quote}
   \textbf{Note,} however, that these \TeX documents are not tracked for changes
   from your \texttt{Makefile}.
\end{quote}

!`\TeX\ can g\`{e}n\'{e}rate alm\"{o}st all the accents\footnote{This is an
example of a footnote.} and spe\c{c}ial symbols used in Western \cite{paper2}
l\aa nguages! Likewise, its arsenal of mathematical sym$\beta$ols, introduced
below, is formidable.

\subsubsection{Mathematical Formulae}
\TeX\ is good at typesetting mathematical formulas like
       $ x-3y = 7 $
or
       $ y_{i+1} = x_{i}^{2n} - \sqrt{5}x_{i}^{n} + 1$.
Remember that a letter like
       \( x \)        % $ ... $  and  \( ... \)  are equivalent
is a formula when it denotes a mathematical symbol, and should
be treated as one.

Mathematical formulas may also be       %  use \[ \]
{\em displayed}.  A displayed formula       %  or  $$  $$
is one-line long; multiline formulas        %  or \begin{equation}
require special formatting instructions.    %  \begin{eqnarray}
The following formulae demonstrate
many constructions you might find useful.
Refer to equation (\ref{eq:fermat}), which is probably true,
while equations (\ref{eq:dumb}-\ref{eq:realdumb}) are silly.
Note that the \verb9equation9 and \verb9eqnarray9 environments
number the equations, but \verb9eqnarray*9 doesn't.
\[  x_{i+1} ~=~ N^{i+1}(x_{0}) ~=~ N(x_{i}) ~=~
    x_{i} - \frac{f(x_{i})}{f'(x_{i})}  \]

$$ \frac{\partial u}{\partial t} + \nabla^{4}u + \nabla^{2}u +
    \frac12    |\nabla u|^{2}~ =~ c^2   $$

\begin{equation}  a^{p} + b^p   \neq c^{p} ~~~\mbox{for } p>2
    ~~ \mbox{(see proof in margin)}  \label{eq:fermat}
\end{equation}
$$ \lim_{n \rightarrow \infty}x_{n} \geq \pi $$
$$ \forall x \in {\cal O} ~~\exists \delta ~~~\mbox{such that}~~~
    |y-x|<\delta ~\Rightarrow ~y \in {\cal O} $$
\vspace{2mm}
$$
\Psi' = \frac{d}{d \phi} \left( \begin{array}{c}
  \phi_{2}  \\  \phi_{3}  \\  1 - \phi_{2} - \phi_{1}^{2}/2
 \end{array} \right)
 ~~~~~~~~~~~~~
 \Theta =  \left(   \begin{array}{ccc}
    0           &   1   &   0   \\
- \theta_{1} \psi_{1} - \psi_{2} &  0   &   \psi_3  \\
    -\phi_{1}           &   -1  &   0
            \end{array}  \right)
$$
\vspace{2mm}
\begin{eqnarray}
    \int_0^{\infty} e^{-x^2}\,dx
    & = &   e^{-\left(\int_0^{\infty}x\,dx\right)^2} \label{eq:dumb}  \\
    & = &   e^{-\infty} ~~~~~\mbox{(bogus)} \\
    & = &   0.38-1.7i ~~~~~\mbox{(not!)} \label{eq:realdumb}
\end{eqnarray}
\vspace{2mm}
\begin{eqnarray*}           %  "*" = no line numbering
  \sum_{n=1}^k \frac1n
    & \approx & \ln k + \gamma  \\
    & = &       (\ln 10)(\log_{10}k) + \gamma \\
    & \approx & 2.3026\log_{10}k + 0.57772
\end{eqnarray*}

Unary operators ``plus'' and ``minus'' -- just use exponentiation:
$$  {}^{+}0.168  \mbox{ or } {}^{-}1.168    $$


\section{Use of pure \LaTeX}

You can integrate a \LaTeX{} file into this report.
To do this, include it with the command \texttt{\textbackslash{}input\{content/filename.tex\}} into one of your (R)md-documents.

\begin{quote}
   \textbf{Note,} however, that these \TeX documents are not tracked for changes
   from your \texttt{Makefile}.
\end{quote}

!`\TeX\ can g\`{e}n\'{e}rate alm\"{o}st all the accents\footnote{This is an
example of a footnote.} and spe\c{c}ial symbols used in Western \cite{paper2}
l\aa nguages! Likewise, its arsenal of mathematical sym$\beta$ols, introduced
below, is formidable.

\subsubsection{Mathematical Formulae}
\TeX\ is good at typesetting mathematical formulas like
       $ x-3y = 7 $
or
       $ y_{i+1} = x_{i}^{2n} - \sqrt{5}x_{i}^{n} + 1$.
Remember that a letter like
       \( x \)        % $ ... $  and  \( ... \)  are equivalent
is a formula when it denotes a mathematical symbol, and should
be treated as one.

Mathematical formulas may also be       %  use \[ \]
{\em displayed}.  A displayed formula       %  or  $$  $$
is one-line long; multiline formulas        %  or \begin{equation}
require special formatting instructions.    %  \begin{eqnarray}
The following formulae demonstrate
many constructions you might find useful.
Refer to equation (\ref{eq:fermat}), which is probably true,
while equations (\ref{eq:dumb}-\ref{eq:realdumb}) are silly.
Note that the \verb9equation9 and \verb9eqnarray9 environments
number the equations, but \verb9eqnarray*9 doesn't.
\[  x_{i+1} ~=~ N^{i+1}(x_{0}) ~=~ N(x_{i}) ~=~
    x_{i} - \frac{f(x_{i})}{f'(x_{i})}  \]

$$ \frac{\partial u}{\partial t} + \nabla^{4}u + \nabla^{2}u +
    \frac12    |\nabla u|^{2}~ =~ c^2   $$

\begin{equation}  a^{p} + b^p   \neq c^{p} ~~~\mbox{for } p>2
    ~~ \mbox{(see proof in margin)}  \label{eq:fermat}
\end{equation}
$$ \lim_{n \rightarrow \infty}x_{n} \geq \pi $$
$$ \forall x \in {\cal O} ~~\exists \delta ~~~\mbox{such that}~~~
    |y-x|<\delta ~\Rightarrow ~y \in {\cal O} $$
\vspace{2mm}
$$
\Psi' = \frac{d}{d \phi} \left( \begin{array}{c}
  \phi_{2}  \\  \phi_{3}  \\  1 - \phi_{2} - \phi_{1}^{2}/2
 \end{array} \right)
 ~~~~~~~~~~~~~
 \Theta =  \left(   \begin{array}{ccc}
    0           &   1   &   0   \\
- \theta_{1} \psi_{1} - \psi_{2} &  0   &   \psi_3  \\
    -\phi_{1}           &   -1  &   0
            \end{array}  \right)
$$
\vspace{2mm}
\begin{eqnarray}
    \int_0^{\infty} e^{-x^2}\,dx
    & = &   e^{-\left(\int_0^{\infty}x\,dx\right)^2} \label{eq:dumb}  \\
    & = &   e^{-\infty} ~~~~~\mbox{(bogus)} \\
    & = &   0.38-1.7i ~~~~~\mbox{(not!)} \label{eq:realdumb}
\end{eqnarray}
\vspace{2mm}
\begin{eqnarray*}           %  "*" = no line numbering
  \sum_{n=1}^k \frac1n
    & \approx & \ln k + \gamma  \\
    & = &       (\ln 10)(\log_{10}k) + \gamma \\
    & \approx & 2.3026\log_{10}k + 0.57772
\end{eqnarray*}

Unary operators ``plus'' and ``minus'' -- just use exponentiation:
$$  {}^{+}0.168  \mbox{ or } {}^{-}1.168    $$
